% This file includes various thermodynamic shapes, usable in regular TikZ graphics.


\tikzset{
    symmetrycross/.style={%
        % Style for a center-cross to mark symmetry in technical drawings.
        % Styles have one argument per default. Use it for size,
        % but also provide a default
        draw,
        cross out,% From shapes library
        thin,
        dashdotted,
        rotate=45,
        minimum size=#1,% This is an argument
    },
    symmetrycross/.default={2em},
    pipecrosssection/.style={
        circle,
        minimum size=#1,
        draw=black!90!red,
        ultra thick,% Simulate thick pipe around it
        fill=MediumFluid
    },
    pipecrosssection/.default={0.5em},
    arrowlabel/.style={ % Style for labels along arrows
        pos=#1,
        text=black,% Overwrite in case encapsulating draw command is white etc.
        draw=none,% Do not draw border
        fill=white,
        inner sep=1.5pt,
        rounded corners,
    },
    arrowlabel/.default={0.5}, % Default for position: middle
    wall/.style={ % For all walls, e.g. in crosssections of pipes, ...
        draw,
        thick,
        fill=#1,
        line join=round,
        rounded corners=0.5,
    },
    wall/.default={g5},
    fluid/.style={ % For all fluids
        fill=#1,
    },
    fluid/.default={MediumFluid},
    flowarrow/.style 2 args={% A style for squiggly arrows, for flows:
        ->,
        line join=round,
        thick,
        dashed,
        decorate,
        decoration={
            snake, % alternatives: snake, zigzag
            segment length=#1,
            amplitude=#2,
            % Stop with decoration at start and end by the given length and just draw
            % a line:
            pre=lineto,
            post=lineto,
            pre length=1em,
            post length=1em
        },
    },
    annotationarrow/.style={%
        % A style intended to point a line with a thick black dot at its end onto
        % elements to give further explanations.
        % Since we shorten the line START, any line using this style should originate
        % from the place where the shortening is supposed to take place.
        % Further explanation here:
        % https://tex.stackexchange.com/a/115268/120853
        % Simply call -* (or any other similar command) after annoationarrow if the
        % direction should be inverted.
        *-,
        shorten <=-(1.8pt + 1.4\pgflinewidth),%
    },
    flowarrow/.default={3em}{2em},% {<INVERSE FREQUENCY>}{<AMPLITUDE>}
    origindot/.style={ % A simple small dot for coordinate starts/origins
        circle,
        fill=#1,
        inner sep=0pt,
        minimum width=0.4em
    },
    origindot/.default={black},
    add node at x/.style 2 args={% https://tex.stackexchange.com/a/93968/120853
        % Add node on a plot by just specifying its x-position.
        % The corresponding y/value is computed automatically!
        name path global=plot line,
        /pgfplots/execute at end plot visualization/.append={
                \begingroup
                \@ifnextchar[{\parsenode}{\parsenode[]}#2\pgf@nil
            \path [name path global=position line #1-1]
                ({axis cs:#1,0}|-{rel axis cs:0,0}) --
                ({axis cs:#1,0}|-{rel axis cs:0,1});
            \path [xshift=1pt, name path global=position line #1-2]
                ({axis cs:#1,0}|-{rel axis cs:0,0}) --
                ({axis cs:#1,0}|-{rel axis cs:0,1});
            \path [
                name intersections={
                    of={plot line and position line #1-1},
                    name=left intersection
                },
                name intersections={
                    of={plot line and position line #1-2},
                    name=right intersection
                },
                label node/.append style={pos=1}
            ] (left intersection-1) -- (right intersection-1)
            node [label node]{\nodetext};
            \endgroup
        }
    },
    shorten <>/.style={% Allow shortening of both ends simultaneously
        shorten >=#1,
        shorten <=#1,
    },
    circlednum/.style={% Used for a custom circled number command
        shape=circle,
        draw,
        inner sep=2pt,
        fill=white
    }%,
    % A reusable little 'pic' for a common radiator.
    % Further down, we also declare a 'shape' for a radiator. A 'shape' is a node shape with anchors etc. and intended to be used to connect nodes together, like in a circuit diagram.
    % A 'pic' is really just a reusable picture that can be put anywhere, but it (by the package author) not intended to be connected etc.
    % This pic is prettier than the shape below. The pic should be used for large radiator images where more detail is desired.
    radiator/.pic={
        \tikzset{%
            every path/.style={
                    draw,
                    thick,
                    line join=round,
                    line cap=round,% End of lines
                    fill=g5,
                    pic actions,% Whatever is used in \pic[<drawing options>], e.g. dashed, is propagated to this drawing
                },
        }
        % Base measurements. Default unit is centimeter
        \pgfmathsetmacro{\radiatorheight}{2}
        \pgfmathsetmacro{\radiatorwidth}{3}
        \pgfmathsetmacro{\radiatordepth}{0.5}

        % Right surface:
        \begin{scope}[canvas is yz plane at x=\radiatorwidth]
            \draw[fill=g4] (0,0) rectangle (\radiatorheight,\radiatordepth);

            % Fake the thermostat a bit:
            \draw[fill=g3] (0.8*\radiatorheight,0.5*\radiatordepth) circle (0.2*\radiatordepth);

            % Generate some coordinates in analogy to regular tikz naming convetions:
            \foreach \yfrac/\zfrac/\coordname in {%
                    0.5/0.5/center,%
                    0/0.5/south,%
                    1/0.5/north,%
                    0.5/0/east,
                    0.5/1/west,
                    0/0/south east,%
                    0/1/south west,%
                    1/1/north east,%
                    1/0/north west,%
                    0.8/0.5/thermostat%
                }{
                    \coordinate (-right \coordname) at (\yfrac*\radiatorheight,\zfrac*\radiatordepth);
                }
        \end{scope}

        % Top surface:
        \begin{scope}[canvas is xz plane at y=\radiatorheight]
            \draw (0,0) rectangle (\radiatorwidth,\radiatordepth);

            % "Hollow space" inside the radiator:
            \draw[fill=g2, rounded corners=0.75] (0.05*\radiatorwidth,0.3*\radiatordepth) rectangle (0.95*\radiatorwidth,0.7*\radiatordepth);

            % Generate some coordinates in analogy to regular tikz naming convetions:
            \foreach \xfrac/\zfrac/\coordname in {%
                    0.5/0.5/center,%
                    0.5/1/south,%
                    0.5/0/north,%
                    1/0.5/east,
                    0/0.5/west,
                    1/1/south east,%
                    0/1/south west,%
                    1/0/north east,%
                    0/0/north west%
                }{
                    \coordinate (-top \coordname) at (\xfrac*\radiatorwidth,\zfrac*\radiatordepth);
                }
        \end{scope}

        % Front surface:
        \begin{scope}[canvas is xy plane at z=\radiatordepth]
            \draw (0,0) rectangle (\radiatorwidth,\radiatorheight);

            \foreach \posfraction in {0.1, 0.2, ..., 0.9}
                {%
                    \draw[
                        thick,
                        preaction={draw, ultra thick, g3},
                    ] (\posfraction*\radiatorwidth, 0.1*\radiatorheight) -- (\posfraction*\radiatorwidth, 0.9*\radiatorheight);
                }
            % Generate some coordinates in analogy to regular tikz naming convetions:
            \foreach \xfrac/\yfrac/\coordname in {%
                    0.5/0.5/center,%
                    0.5/0/south,%
                    0.5/1/north,%
                    1/0.5/east,
                    0/0.5/west,
                    1/0/south east,%
                    0/0/south west,%
                    1/1/north east,%
                    0/1/north west%
                }{
                    \coordinate (-front \coordname) at (\xfrac*\radiatorwidth,\yfrac*\radiatorheight);
                }
            \node at (-front center) {\ctrw{\tikzpictext}};% value of "pic text" key is stored in \tikzpictext macro
        \end{scope}

        % Coordinates for convenience:
        \coordinate (-thermostat) at (-right thermostat);
        \coordinate (-entry) at ($(-front north west)!0.2!(-front south west)$);
        \coordinate (-exit) at ($(-right south)!0.2!(-right north)$);
        \coordinate (-left center) at (-front west|--right center);% To have an exact opposite of the visible 'center right' position

        % Debugging coordinate system:
        % \draw[->] (0,0,0) -- (1,0,0) node[right] {\(x\)};
        % \draw[->] (0,0,0) -- (0,1,0) node[left] {\(y\)};
        % \draw[->] (0,0,0) -- (0,0,1) node[above] {\(z\)};
    },
    TUHHuman/.pic={% A human member of TUHH. Hilarious......
        \pgfmathsetmacro{\roundingradius}{0.4}
        \filldraw[pic actions] (0,0) circle (1);% Head
        %
        % Use arc over just specifying 'rounded corners', because rounded corners are absolute and aren't affected by 'scale'
        \filldraw[pic actions]
        (0,-1.2)% Below head
        -- ++ (2.5-\roundingradius,0) arc (90:0:\roundingradius)% Right shoulder
        -- ++ (0,{-(5 - 2*\roundingradius)}) arc (0:-90:\roundingradius) -- ++ ({-(0.8 - 2*\roundingradius)},0) arc (270:180:\roundingradius) -- ++ (0,{4 - \roundingradius})% Right arm
        -- ++ (-0.25,0)% Right arm pit
        -- ++ (0,{-(8 - \roundingradius)}) arc (0:-90:\roundingradius) -- ++ ({-(1 - 2*\roundingradius)},0) arc (270:180:\roundingradius) -- ++ (0,{4 - 2*\roundingradius}) arc (0:90:\roundingradius)% Right leg
        -- ++ ({-(1 - 2*\roundingradius)},0)% Gap between legs
        arc (90:180:\roundingradius) -- ++ (0,{-(4 - 2*\roundingradius)}) arc (0:-90:\roundingradius) -- ++ ({-(1 - 2*\roundingradius)},0) arc (270:180:\roundingradius) -- ++ (0,{8 - \roundingradius})% Left leg
        -- ++ (-0.25,0)% Left arm pit
        -- ++ (0,{-(4 - \roundingradius)}) arc (0:-90:\roundingradius) -- ++ ({-(0.8 - 2*\roundingradius},0) arc (270:180:\roundingradius) -- ++ (0,{5 - 2*\roundingradius})% Left arm
        arc (180:90:\roundingradius)-- cycle;% Left shoulder; cycle back to below head

        \node[right, transform shape] at (-1.25,-2.5) {\ctrw{\textcolor[RGB]{45,198,214}{\textbf{\textsf{TUHH}}}}};% Chest emblem; transform shape allows node to be affected by 'scale' option
        %
        % \draw[help lines] (-3,-11) grid (3,1);% Debugging grid
    },
    boilercylinder/.style={
        cylinder,% Shapes lib
        draw,
        thick,
        shape border rotate=90,
        aspect=0.3,% Flatten
        cylinder uses custom fill,
        % cylinder body fill=g5,% For regular filling, use these and comment out the colors at the bottom here
        % cylinder end fill=g4,%
        left color=HotFluid!70,
        middle color=orange!20,
        right color=g4,
        shading angle=180,
    },
    % Still use a pic so we can flexible add stuff to all boilers later one, like a burner on the bottom left side:
    boiler/.pic={
        \node[
            boilercylinder,
            minimum width=4em,
            minimum height=6em,
            pic actions,% Whatever is used in \pic[<drawing options>], e.g. dashed, is propagated to this drawing
        ] () {\ctrw{\tikzpictext}};% Empty coordinate is important for namespace stuff
    },
    equalizing tank/.style={
        circle split,
        draw,
        solid,% In case surrounding context is dashed etc.
        -,% As opposed to arrows etc.
        fill=white,
        double,
        minimum width=#1,
    },
    equalizing tank/.default={1.5em},
    pipe/.style={% (Water) pipes between devices. For a pipe without arrow line end, call it as e.g. \draw[pipe, -]
        line width=#1,
        double=MediumFluid,
        double distance={1*#1},
        -{Triangle Cap []. Fast Triangle[] Fast Triangle[] Fast Triangle[]},% dot represents line end
        line join=round,
    },
    pipe/.default={1.5pt},
    valve/.style={
        shape=valve,
        draw,
        solid,% In case surrounding context is dashed etc.
        -,% As opposed to arrows etc.
        fill=white,
        rounded corners=0.5,
        rotate=90,% We probably need the horizontal version more often. Also, this way, \draw ... node[midway, valve, sloped] {}; works out of the box
        minimum width=0.6*#1,% Fixed aspect ratio
        minimum height=#1,
    },
    valve/.default={1em},
    control valve/.style={% Same node style but different shape
        valve,
        % blue!10,
        shape=controlvalve,
    },
    threeway valve/.style={
        valve,
        shape=threeway valve,
        minimum width=0.85*#1,% Scale it a bit...
        minimum height=#1,
    },
    threeway valve/.default={1em},
    fourway valve/.style={
        valve,
        rotate=-90,% It's symmetrical, no need to rotate; undo inherited rotation
        shape=fourway valve,
        minimum width=#1,% Fixed aspect ratio, square
        minimum height=#1,
    },
    fourway valve/.default={1em},
    threeway control valve/.style={
        threeway valve,
        shape=threewaycontrolvalve,
    },
    radiator/.style={
        shape=radiator,
        draw,
        fill=g6,
        line join=round,
        minimum width=#1,
        minimum height=(2/3)*#1,% Fixed aspect ratio
    },
    radiator/.default={3em},
    vented radiator/.style={
        radiator,% Inherit from style
        shape=vented radiator,% Different shape
    },
    pump/.style={
        shape=pump,
        draw,
        solid,% In case surrounding context is dashed etc.
        -,% As opposed to arrows etc.
        fill=white,
        line join=round,
        minimum width=#1,
        rounded corners=0.5,% Alter rounded corners in case surrounding style has them
    },
    pump/.default={1.5em},
    compressor/.style={
        shape=compressor,
        draw,
        solid,
        -,
        fill=white,
        line join=round,
        minimum size=#1,
        sloped,% Because why not
    },
    compressor/.default={1.5em},
    heat exchanger/.style={
        shape=heat exchanger,
        draw,
        solid,
        -,
        fill=white,
        line join=round,
        rounded corners=0.5,
        minimum width=#1,
        minimum height=0.5*#1,% Fixed aspect ratio
        sloped,% Because why not
    },
    heat exchanger/.default={1.5em},
    simple heat exchanger/.style={
        shape=simple heat exchanger,
        draw,
        solid,
        -,
        fill=white,
        line join=round,
        minimum width=#1,
        minimum height=2/3*#1,% Fixed aspect ratio
        sloped,% Because why not
    },
    simple heat exchanger/.default={1.5em},
    sensor/.style={% A vertical stroke with with three horizontal lines of decreasing width underneath
        shape=sensor,
        draw,
        solid,
        -,
        line join=round,
        line cap=round,
        minimum width=#1,
        minimum height=2*#1
    },
    sensor/.default={0.5em},
    level indicator/.style={% An upside-down triangle with three horizontal lines of decreasing width underneath
        shape=levelindicator,
        draw,
        fill=white,
        solid,
        -,
        line join=round,
        line cap=round,
        minimum width=#1,
        minimum height=2*#1,
        inner sep=1.5pt,
    },
    level indicator/.default={0.25em}
}

\pgfdeclareshape{valve}{%
    \inheritsavedanchors[from=rectangle]% Inherit from rectangle, should be good enough
    \inheritanchorborder[from=rectangle]
    \inheritanchor[from=rectangle]{north}
    \inheritanchor[from=rectangle]{north west}
    \inheritanchor[from=rectangle]{north east}
    \inheritanchor[from=rectangle]{center}
    \inheritanchor[from=rectangle]{west}
    \inheritanchor[from=rectangle]{east}
    \inheritanchor[from=rectangle]{mid}
    \inheritanchor[from=rectangle]{mid west}
    \inheritanchor[from=rectangle]{mid east}
    \inheritanchor[from=rectangle]{base}
    \inheritanchor[from=rectangle]{base west}
    \inheritanchor[from=rectangle]{base east}
    \inheritanchor[from=rectangle]{south}
    \inheritanchor[from=rectangle]{south west}
    \inheritanchor[from=rectangle]{south east}

    \backgroundpath{%
        % Store lower left in xa/ya and upper right in xb/yb
        \southwest \pgf@xa=\pgf@x \pgf@ya=\pgf@y
        \northeast \pgf@xb=\pgf@x \pgf@yb=\pgf@y

        % Construct main path, basically 'two triangles touching'
        \pgfpathmoveto{\pgfpoint{\pgf@xa}{\pgf@ya}}
            \pgfpathlineto{\pgfpoint{\pgf@xb}{\pgf@yb}}
            \pgfpathlineto{\pgfpoint{\pgf@xa}{\pgf@yb}}
            \pgfpathlineto{\pgfpoint{\pgf@xb}{\pgf@ya}}
        \pgfpathclose
    }
}

\pgfdeclareshape{threeway valve}{
    % This is asymetrical and stuff, so inheriting from the base valve is a bit hopeless, I don't know how to do it properly

    \inheritsavedanchors[from=rectangle]% Inherit from rectangle, should be good enough
    \inheritanchorborder[from=rectangle]

    \inheritanchor[from=rectangle]{north west}
    \inheritanchor[from=rectangle]{north east}
    \inheritanchor[from=rectangle]{center}
    \inheritanchor[from=rectangle]{west}
    \inheritanchor[from=rectangle]{east}
    \inheritanchor[from=rectangle]{mid}
    \inheritanchor[from=rectangle]{mid west}
    \inheritanchor[from=rectangle]{mid east}
    \inheritanchor[from=rectangle]{base}
    \inheritanchor[from=rectangle]{base west}
    \inheritanchor[from=rectangle]{base east}
    \inheritanchor[from=rectangle]{south west}
    \inheritanchor[from=rectangle]{south east}

    % These have shifted now, since the three-point valve middle is not the node center anymore:
    \anchor{center}{
        \northeast \pgf@xb=\pgf@x \pgf@yb=\pgf@y
        \southwest \pgf@xa=\pgf@x \pgf@ya=\pgf@y
        \pgfmathsetlength\pgf@x{\pgf@xa + 0.5*0.75*(\pgf@xb - \pgf@xa)}
        \pgfmathsetlength\pgf@y{\pgf@ya + 0.5*(\pgf@yb - \pgf@ya)}
    }
    \anchor{north}{
        \northeast \pgf@xb=\pgf@x \pgf@yb=\pgf@y
        \southwest \pgf@xa=\pgf@x
        \pgfmathsetlength\pgf@x{\pgf@xa + 0.5*0.75*(\pgf@xb - \pgf@xa)}
        \pgfmathsetlength\pgf@y{\pgf@yb}
    }
    \anchor{south}{
        \northeast \pgf@xb=\pgf@x
        \southwest \pgf@xa=\pgf@x \pgf@ya=\pgf@y
        \pgfmathsetlength\pgf@x{\pgf@xa + 0.5*0.75*(\pgf@xb - \pgf@xa)}
        \pgfmathsetlength\pgf@y{\pgf@ya}
    }

    % The triangles meeting at their middle, each with an aspect ratio of 0.75 (base) to 0.5 (height)
    \backgroundpath{
        \southwest \pgf@xa=\pgf@x \pgf@ya=\pgf@y
        \northeast \pgf@xb=\pgf@x \pgf@yb=\pgf@y

        \pgfpathmoveto{\southwest}
        \pgfpathlineto{\pgfpoint{\pgf@xa + 0.75*(\pgf@xb - \pgf@xa)}{\pgf@yb}}
        \pgfpathlineto{\pgfpoint{\pgf@xa}{\pgf@yb}}
        \pgfpathlineto{\pgfpoint{\pgf@xa + 0.75*0.5*(\pgf@xb - \pgf@xa)}{\pgf@ya + 0.5*(\pgf@yb - \pgf@ya)}}% Valve centerpoint
        \pgfpathlineto{\pgfpoint{\pgf@xb}{\pgf@ya + 0.8*(\pgf@yb - \pgf@ya)}}
        \pgfpathlineto{\pgfpoint{\pgf@xb}{\pgf@ya + 0.2*(\pgf@yb - \pgf@ya)}}
        \pgfpathlineto{\pgfpoint{\pgf@xa + 0.75*0.5*(\pgf@xb - \pgf@xa)}{\pgf@ya + 0.5*(\pgf@yb - \pgf@ya)}}% Valve centerpoint
        \pgfpathlineto{\pgfpoint{\pgf@xa + 0.75*(\pgf@xb - \pgf@xa)}{\pgf@ya}}
        \pgfpathclose
    }
}

\pgfdeclareshape{fourway valve}{%
    \inheritsavedanchors[from=rectangle]% Inherit from rectangle, should be good enough
    \inheritanchorborder[from=rectangle]
    \inheritanchor[from=rectangle]{north}
    \inheritanchor[from=rectangle]{north west}
    \inheritanchor[from=rectangle]{north east}
    \inheritanchor[from=rectangle]{center}
    \inheritanchor[from=rectangle]{west}
    \inheritanchor[from=rectangle]{east}
    \inheritanchor[from=rectangle]{mid}
    \inheritanchor[from=rectangle]{mid west}
    \inheritanchor[from=rectangle]{mid east}
    \inheritanchor[from=rectangle]{base}
    \inheritanchor[from=rectangle]{base west}
    \inheritanchor[from=rectangle]{base east}
    \inheritanchor[from=rectangle]{south}
    \inheritanchor[from=rectangle]{south west}
    \inheritanchor[from=rectangle]{south east}

    \backgroundpath{%
        % Store lower left in xa/ya and upper right in xb/yb
        \southwest \pgf@xa=\pgf@x \pgf@ya=\pgf@y
        \northeast \pgf@xb=\pgf@x \pgf@yb=\pgf@y

        % This yields four triangles meeting in the middle, each with an aspect ratio of 0.6/1.
        % Construct main path, basically 'two triangles touching'
        \pgfpathmoveto{\pgfpoint{\pgf@xa + 0.2*(\pgf@xb - \pgf@xa)}{\pgf@ya}}
            \pgfpathlineto{\pgfpoint{\pgf@xa + 0.8*(\pgf@xb - \pgf@xa)}{\pgf@yb}}
            \pgfpathlineto{\pgfpoint{\pgf@xa + 0.2*(\pgf@xb - \pgf@xa)}{\pgf@yb}}
            \pgfpathlineto{\pgfpoint{\pgf@xa + 0.8*(\pgf@xb - \pgf@xa)}{\pgf@ya}}
        \pgfpathclose

        % Do the same again in the horizontal direction:
        \pgfpathmoveto{\pgfpoint{\pgf@xa}{\pgf@ya + 0.2*(\pgf@yb - \pgf@ya)}}
            \pgfpathlineto{\pgfpoint{\pgf@xa}{\pgf@ya + 0.8*(\pgf@yb - \pgf@ya)}}
            \pgfpathlineto{\pgfpoint{\pgf@xb}{\pgf@ya + 0.2*(\pgf@yb - \pgf@ya)}}
            \pgfpathlineto{\pgfpoint{\pgf@xb}{\pgf@ya + 0.8*(\pgf@yb - \pgf@ya)}}
        \pgfpathclose
    }
}

\pgfdeclareshape{controlvalve}{
    \inheritsavedanchors[from=valve]
    \inheritanchorborder[from=valve]
    \inheritanchor[from=valve]{north}
    \inheritanchor[from=valve]{north west}
    \inheritanchor[from=valve]{north east}
    \inheritanchor[from=valve]{center}
    \inheritanchor[from=valve]{west}
    \inheritanchor[from=valve]{east}
    \inheritanchor[from=valve]{mid}
    \inheritanchor[from=valve]{mid west}
    \inheritanchor[from=valve]{mid east}
    \inheritanchor[from=valve]{base}
    \inheritanchor[from=valve]{base west}
    \inheritanchor[from=valve]{base east}
    \inheritanchor[from=valve]{south}
    \inheritanchor[from=valve]{south west}
    \inheritanchor[from=valve]{south east}

    \inheritbackgroundpath[from={valve}]

    \beforebackgroundpath{%
        % Store lower left in xa/ya and upper right in xb/yb
        \southwest \pgf@xa=\pgf@x \pgf@ya=\pgf@y
        \northeast \pgf@xb=\pgf@x \pgf@yb=\pgf@y

        % Extension line to control wheel:
        \pgfpathmoveto{\pgfpoint{(\pgf@xa + \pgf@xb)/2}{(\pgf@ya + \pgf@yb)/2}}
        \pgfpathlineto{\pgfpoint{\pgf@xb}{(\pgf@ya + \pgf@yb)/2}}

        % The factors 0.3, 0.4 and 0.3 need to add up to 1 to give a symmetrical, nice semicircle.
        % Factor 0.4 is divided by 2 to give the radius
        \pgfpathmoveto{\pgfpoint{\pgf@xb}{\pgf@ya + 0.3*(\pgf@yb - \pgf@ya)}}
        \pgfpatharc{-90}{90}{0.4/2*(\pgf@yb - \pgf@ya)}
        \pgfpathclose
        \pgfsetfillcolor{black}
        \pgfusepath{fill, stroke}
    }
}

\pgfdeclareshape{pump}{%
    \inheritsavedanchors[from=circle]%
    \inheritanchorborder[from=circle]
    \inheritbackgroundpath[from={circle}]
    \inheritanchor[from=circle]{center}
    \inheritanchor[from=circle]{north}
    \inheritanchor[from=circle]{south}
    \inheritanchor[from=circle]{west}
    \inheritanchor[from=circle]{east}

    \beforebackgroundpath{
        % The simple-most approach:
        \pgfpathmoveto{\pgf@anchor@pump@west}
            \pgfpathlineto{\pgf@anchor@pump@north}
            \pgfpathlineto{\pgf@anchor@pump@east}
        \pgfusepath{stroke}
    }
}

\pgfdeclareshape{compressor}{%
    \inheritsavedanchors[from=circle]%
    \inheritanchorborder[from=circle]%
    \inheritbackgroundpath[from={circle}]%
    \inheritanchor[from=circle]{center}%
    \inheritanchor[from=circle]{north}%
    \inheritanchor[from=circle]{south}%
    \inheritanchor[from=circle]{west}%
    \inheritanchor[from=circle]{east}%

    % Circle has savedanchor of 'centerpoint' and saveddim of 'radius', that's it

    \beforebackgroundpath{
        \centerpoint \pgf@yc=\pgf@y

        % Pythagorean theorem is enough here.
        % Assumption is that default compressor shape 'points to the right'.
        % This is important to get coorect behaviour out of the 'sloped' node option!

        % Idea (Example for top right point):
        % 1. Go to center point.
        % 2. On y-axis, move a fraction of the radius up. This is the y-portion of our top-right coordinate point.
        % 3. On x-axis, move as far right as Pythagorean theorem dictates (hypotenuse is the radius vector, other known length is the y-length we just specified)
        % 4. Repeat process to draw and move accordingly; don't forget squares and sqrt() in correct places.

        \pgfmathsetmacro{\upperyfraction}{0.3}% y-coordinate of the upper points is \upperyfraction*\radius to the sides of the centerpoint
        \pgfmathsetmacro{\loweryfraction}{0.7}% y-coordinate of the lower points is \loweryfraction*\radius to the sides of the centerpoint

        \foreach \signdirection in {1, -1}{% For left and right side
            % Upper (always same x-dimension, only y switches sign):
            \pgfpathmoveto{\pgfpoint{sqrt(\radius^2 - (\pgf@yc + \upperyfraction*\radius)^2)}{\pgf@yc + \signdirection*\upperyfraction*\radius}}
            % Bottom (always same x-dimension, only y switches sign):
            \pgfpathlineto{\pgfpoint{-1*sqrt(\radius^2 - (\pgf@yc - \loweryfraction*\radius)^2)}{\pgf@yc + \signdirection*\loweryfraction*\radius}}
        }

        \pgfusepath{stroke}
    }
}

\pgfdeclareshape{heat exchanger}{
    \inheritsavedanchors[from=rectangle]% Inherit from rectangle, should be good enough. Rectangle has to 'saved anchors': southwest and northeast
    \inheritanchorborder[from=rectangle]
    % Doesn't hurt to also inherit specific anchors:
    \inheritanchor[from=rectangle]{north}
    \inheritanchor[from=rectangle]{north west}
    \inheritanchor[from=rectangle]{north east}
    \inheritanchor[from=rectangle]{center}
    \inheritanchor[from=rectangle]{west}
    \inheritanchor[from=rectangle]{east}
    \inheritanchor[from=rectangle]{mid}
    \inheritanchor[from=rectangle]{mid west}
    \inheritanchor[from=rectangle]{mid east}
    \inheritanchor[from=rectangle]{base}
    \inheritanchor[from=rectangle]{base west}
    \inheritanchor[from=rectangle]{base east}
    \inheritanchor[from=rectangle]{south}
    \inheritanchor[from=rectangle]{south west}
    \inheritanchor[from=rectangle]{south east}

    \inheritbackgroundpath[from={rectangle}]

    % Anchor to 'in' port of heat exchanger.
    % This code does not get executed necessarily; we cannot use the \pgfsetmacros from below.
    % A solution is probably \pgfkeys, but this will do for now.
    % IT NEEDS MANUAL ADJUSTING IF THE SHAPE IS CHANGED.
    \anchor{in}{%
        \southwest \pgf@xa=\pgf@x \pgf@ya=\pgf@y
        \northeast \pgf@xb=\pgf@x \pgf@yb=\pgf@y
        \pgfmathsetlength\pgf@x{\pgf@xa + (1 - 0.7)/2*(\pgf@xb - \pgf@xa)}
        \pgfmathsetlength\pgf@y{\pgf@ya - 0.3*(\pgf@yb - \pgf@ya)}
    }
    \anchor{out}{%
        \southwest \pgf@xa=\pgf@x \pgf@ya=\pgf@y
        \northeast \pgf@xb=\pgf@x \pgf@yb=\pgf@y
        \pgfmathsetlength\pgf@x{\pgf@xb - (1 - 0.7)/2*(\pgf@xb - \pgf@xa)}
        \pgfmathsetlength\pgf@y{\pgf@ya - 0.3*(\pgf@yb - \pgf@ya)}
    }

    \beforebackgroundpath{
        \southwest \pgf@xa=\pgf@x \pgf@ya=\pgf@y
        \northeast \pgf@xb=\pgf@x \pgf@yb=\pgf@y

        \pgfmathsetmacro{\heatsymbolextension}{0.3}% How far out the 'legs' extend out of the rectangle as a fraction of the overall node's height
        \pgfmathsetmacro{\heatsymbolwidth}{0.7}% The inner drawing's width as a fraction of the node's overall width
        \pgfmathsetmacro{\heatsymbolheight}{0.8}% How high the inner drawing reaches as a fraction of the node's overall height (at 1, it touches the upper rectangle border)
        \pgfmathsetmacro{\heatsymbolindentation}{0.3}% How 'indented' the triangle is, i.e. the vertical distance between the inner drawings highest point and the triangle's down-pointing 'tip'

        \pgfpathmoveto{\pgfpoint{\pgf@xa + (1 - \heatsymbolwidth)/2*(\pgf@xb - \pgf@xa)}{\pgf@ya - \heatsymbolextension*(\pgf@yb - \pgf@ya)}}% Bottom left corner
        \pgfpathlineto{\pgfpoint{\pgf@xa + (1 - \heatsymbolwidth)/2*(\pgf@xb - \pgf@xa)}{\pgf@ya + \heatsymbolheight*(\pgf@yb - \pgf@ya)}}% Straight line up
        \pgfpathlineto{\pgfpoint{\pgf@xa + 0.5*(\pgf@xb - \pgf@xa)}{\pgf@ya + (\heatsymbolheight - \heatsymbolindentation)*(\pgf@yb - \pgf@ya)}}% Line down, to the right; x-coordinate is the middle of the node
        \pgfpathlineto{\pgfpoint{\pgf@xb - (1 - \heatsymbolwidth)/2*(\pgf@xb - \pgf@xa)}{\pgf@ya + \heatsymbolheight*(\pgf@yb - \pgf@ya)}}% Line up, to the right
        \pgfpathlineto{\pgfpoint{\pgf@xb - (1 - \heatsymbolwidth)/2*(\pgf@xb - \pgf@xa)}{\pgf@ya - \heatsymbolextension*(\pgf@yb - \pgf@ya)}}% Line straight down
        \pgfusepath{stroke}
    }
}

\pgfdeclareshape{simple heat exchanger}{% Just a cross-out rectangle
    \inheritsavedanchors[from=rectangle]% Inherit from rectangle, should be good enough. Rectangle has to 'saved anchors': southwest and northeast
    \inheritanchorborder[from=rectangle]
    % Doesn't hurt to also inherit specific anchors:
    \inheritanchor[from=rectangle]{north}
    \inheritanchor[from=rectangle]{north west}
    \inheritanchor[from=rectangle]{north east}
    \inheritanchor[from=rectangle]{center}
    \inheritanchor[from=rectangle]{west}
    \inheritanchor[from=rectangle]{east}
    \inheritanchor[from=rectangle]{mid}
    \inheritanchor[from=rectangle]{mid west}
    \inheritanchor[from=rectangle]{mid east}
    \inheritanchor[from=rectangle]{base}
    \inheritanchor[from=rectangle]{base west}
    \inheritanchor[from=rectangle]{base east}
    \inheritanchor[from=rectangle]{south}
    \inheritanchor[from=rectangle]{south west}
    \inheritanchor[from=rectangle]{south east}

    \inheritbackgroundpath[from={rectangle}]

    \beforebackgroundpath{
        \southwest \pgf@xa=\pgf@x \pgf@ya=\pgf@y
        \northeast \pgf@xb=\pgf@x \pgf@yb=\pgf@y

        \pgfpathmoveto{\pgfpoint{\pgf@xa}{\pgf@yb}}% Top left corner
        \pgfpathlineto{\pgfpoint{\pgf@xb}{\pgf@ya}}% Bottom right corner
        \pgfusepath{stroke}
    }
}

\pgfkeys{/pgf/.cd,
    radiator offset x/.initial=0.3em,
    radiator offset y/.initial=0.3em
}
\pgfdeclareshape{radiator}{%
    \inheritsavedanchors[from=rectangle]% Inherit from rectangle, should be good enough. Rectangle has to 'saved anchors': southwest and northeast
    \inheritanchorborder[from=rectangle]
    % Doesn't hurt to also inherit specific anchors:
    \inheritanchor[from=rectangle]{north}
    \inheritanchor[from=rectangle]{north west}
    \inheritanchor[from=rectangle]{north east}
    \inheritanchor[from=rectangle]{center}
    \inheritanchor[from=rectangle]{west}
    \inheritanchor[from=rectangle]{east}
    \inheritanchor[from=rectangle]{mid}
    \inheritanchor[from=rectangle]{mid west}
    \inheritanchor[from=rectangle]{mid east}
    \inheritanchor[from=rectangle]{base}
    \inheritanchor[from=rectangle]{base west}
    \inheritanchor[from=rectangle]{base east}
    \inheritanchor[from=rectangle]{south}
    \inheritanchor[from=rectangle]{south west}
    \inheritanchor[from=rectangle]{south east}

    \saveddimen{\xoffset}{% https://tex.stackexchange.com/a/181283/120853
        \pgfmathsetlength\pgf@x{\pgfkeysvalueof{/pgf/radiator offset x}}
    }
    \saveddimen{\yoffset}{
        \pgfmathsetlength\pgf@y{\pgfkeysvalueof{/pgf/radiator offset y}}
    }

    % In the middle of the right surface:
    \anchor{right}{%
        \southwest \pgf@ya=\pgf@y
        \northeast \pgf@xb=\pgf@x \pgf@yb=\pgf@y
        \pgfmathsetlength\pgf@x{\pgf@xb + \xoffset/2}
        \pgfmathsetlength\pgf@y{(\pgf@ya + \pgf@yb)/2 + \yoffset/2}
    }
    % In the middle of the top surface:
    \anchor{top}{%
        \southwest \pgf@xa=\pgf@x
        \northeast \pgf@xb=\pgf@x \pgf@yb=\pgf@y
        \pgfmathsetlength\pgf@x{(\pgf@xa + \pgf@xb)/2 + \xoffset/2}
        \pgfmathsetlength\pgf@y{\pgf@yb + \yoffset/2}
    }
    % Exit at the thermostat:
    \anchor{exit}{%
        \southwest \pgf@ya=\pgf@y
        \northeast \pgf@xb=\pgf@x \pgf@yb=\pgf@y
        \pgfmathsetlength\pgf@x{\pgf@xb + \xoffset/2}
        \pgfmathsetlength\pgf@y{\pgf@ya + 0.8*(\pgf@yb - \pgf@ya) + \yoffset/2}
    }
    % Lower exit on the right side, below the thermostat:
    \anchor{lower exit}{%
        \southwest \pgf@ya=\pgf@y
        \northeast \pgf@xb=\pgf@x \pgf@yb=\pgf@y
        \pgfmathsetlength\pgf@x{\pgf@xb + \xoffset/2}
        \pgfmathsetlength\pgf@y{\pgf@ya + 0.2*(\pgf@yb - \pgf@ya) + \yoffset/2}
    }
    % Entry in the bottom left somewhere:
    \anchor{lower entry}{%
    \northeast \pgf@yb=\pgf@y
    \southwest \pgf@ya=\pgf@y% x is now given implicitly
    \pgfmathsetlength\pgf@y{\pgf@ya + 0.2*(\pgf@yb - \pgf@ya)}
    }

    % Alternative entry in the top left somewhere:
    \anchor{upper entry}{%
        \northeast \pgf@yb=\pgf@y
        \southwest \pgf@ya=\pgf@y% x is now given implicitly
        \pgfmathsetlength\pgf@y{\pgf@ya + 0.8*(\pgf@yb - \pgf@ya)}
    }

    % Front surface is inherited from rectangle:
    \inheritbackgroundpath[from={rectangle}]

    % Background is inherited already, so add behind background path:
    \behindbackgroundpath{%
        % Store lower left in xa/ya and upper right in xb/yb
        \southwest \pgf@xa=\pgf@x \pgf@ya=\pgf@y
        \northeast \pgf@xb=\pgf@x \pgf@yb=\pgf@y

        % Top surface:
        \pgfpathmoveto{\pgfpoint{\pgf@xa}{\pgf@yb}}% Top left
        \pgfpathlineto{\pgfpointadd{\pgfpoint{\pgf@xa}{\pgf@yb}}{\pgfqpoint{\xoffset}{\yoffset}}}% Top left plus offset vector
        \pgfpathlineto{\pgfpointadd{\northeast}{\pgfqpoint{\xoffset}{\yoffset}}}% Top right (northeast) plus offset vector
        \pgfpathlineto{\northeast}
        \pgfpathclose

        % Access current fillcolor as specified by user
        % Make sides darker than current fill color for some 3D effect
        \colorlet{currentfill}{\tikz@fillcolor}
        \pgfsetfillcolor{currentfill!90!black}
        \pgfusepath{fill, stroke}

        % Right surface:
        \pgfpathmoveto{\northeast}
        \pgfpathlineto{\pgfpointadd{\northeast}{\pgfqpoint{\xoffset}{\yoffset}}}% Top right corner plus offset vector
        \pgfpathlineto{\pgfpointadd{\pgfpoint{\pgf@xb}{\pgf@ya}}{\pgfqpoint{\xoffset}{\yoffset}}}% Bottom right corner plus offset vector
        \pgfpathlineto{\pgfpoint{\pgf@xb}{\pgf@ya}}% Bottom right corner
        \pgfpathclose
        \pgfsetfillcolor{currentfill!80!black}% Even darker than top surface
        \pgfusepath{fill, stroke}

        % Draw an ellipse to fake a thermostat or pipe exit.
        % Position it at 0.8 / 80% height
        % Ellipse axes are confusing and I did not do it properly fully
        \pgfpathellipse{\pgfpointadd{\pgfpoint{\pgf@xb}{\pgf@ya + 0.8*(\pgf@yb - \pgf@ya)}}{\pgfpointscale{0.5}{\pgfqpoint{\xoffset}{\yoffset}}}}{\pgfpoint{0.2*\xoffset}{0.2*\yoffset}}{\pgfpoint{0pt}{1pt}}
        \pgfsetfillcolor{black}
        \pgfusepath{fill}
    }
    % Background is already the inherited rectangle. Draw on top of it with beforebackgroundpath
    \beforebackgroundpath{
        % Store lower left in xa/ya and upper right in xb/yb
        \southwest \pgf@xa=\pgf@x \pgf@ya=\pgf@y
        \northeast \pgf@xb=\pgf@x \pgf@yb=\pgf@y

        % Draw vertical lines on top of front surface:
        \foreach \posfraction in {0.2, 0.4, ..., 0.8}{
            \pgfpathmoveto{\pgfpoint{\pgf@xa + \posfraction*(\pgf@xb - \pgf@xa)}{\pgf@ya + 0.2*(\pgf@yb - \pgf@ya)}}
            \pgfpathlineto{\pgfpoint{\pgf@xa + \posfraction*(\pgf@xb - \pgf@xa)}{\pgf@ya + 0.8*(\pgf@yb - \pgf@ya)}}
            \pgfusepath{stroke}
        }
    }
}

\pgfdeclareshape{vented radiator}{
    \inheritsavedanchors[from=radiator]% Inherit from rectangle, should be good enough. Rectangle has to 'saved anchors': southwest and northeast
    \inheritanchorborder[from=radiator]
    % Doesn't hurt to also inherit specific anchors:
    \inheritanchor[from=radiator]{north}
    \inheritanchor[from=radiator]{north west}
    \inheritanchor[from=radiator]{north east}
    \inheritanchor[from=radiator]{center}
    \inheritanchor[from=radiator]{west}
    \inheritanchor[from=radiator]{east}
    \inheritanchor[from=radiator]{mid}
    \inheritanchor[from=radiator]{mid west}
    \inheritanchor[from=radiator]{mid east}
    \inheritanchor[from=radiator]{base}
    \inheritanchor[from=radiator]{base west}
    \inheritanchor[from=radiator]{base east}
    \inheritanchor[from=radiator]{south}
    \inheritanchor[from=radiator]{south west}
    \inheritanchor[from=radiator]{south east}

    % Inherit our special nodes:
    \inheritanchor[from=radiator]{right}
    \inheritanchor[from=radiator]{top}
    \inheritanchor[from=radiator]{exit}
    \inheritanchor[from=radiator]{lower entry}
    \inheritanchor[from=radiator]{upper entry}

    % Inherit all stuff from base radiator:
    \inheritbackgroundpath[from={radiator}]
    \inheritbehindbackgroundpath[from={radiator}]
    \inheritbeforebackgroundpath[from={radiator}]

    % Anchor for the ventilation point.
    % I tried with \savedanchor, but it didn't work so we have to repeat the calculations for the ventilation device
    \anchor{ventilation}{%
        \southwest \pgf@xa=\pgf@x
        \northeast \pgf@yb=\pgf@y
        \pgfmathsetlength\pgf@x{\pgf@xa - 3pt}
        \pgfmathsetlength\pgf@y{\pgf@yb - 3pt}
    }

    \behindforegroundpath{
        \southwest \pgf@xa=\pgf@x
        \northeast \pgf@yb=\pgf@y

        \pgfpathmoveto{\pgfpoint{\pgf@xa}{\pgf@yb}}% Top left corner
        \pgfpatharc{90}{270}{3pt}% Arc from top-top left corner down with specified radius in last argument
        \pgfsetfillcolor{black}
        \pgfusepath{fill}
    }
}

\pgfdeclareshape{sensor}{
    \inheritsavedanchors[from=rectangle]% Inherit from rectangle, should be good enough. Rectangle has to 'saved anchors': southwest and northeast
    \inheritanchorborder[from=rectangle]
    % Doesn't hurt to also inherit specific anchors:
    \inheritanchor[from=rectangle]{north}
    \inheritanchor[from=rectangle]{north west}
    \inheritanchor[from=rectangle]{north east}
    \inheritanchor[from=rectangle]{center}
    \inheritanchor[from=rectangle]{west}
    \inheritanchor[from=rectangle]{east}
    \inheritanchor[from=rectangle]{south}
    \inheritanchor[from=rectangle]{south west}
    \inheritanchor[from=rectangle]{south east}

    \backgroundpath{
        % Store lower left in xa/ya and upper right in xb/yb
        \southwest \pgf@xa=\pgf@x \pgf@ya=\pgf@y
        \northeast \pgf@xb=\pgf@x \pgf@yb=\pgf@y

        \pgfpathmoveto{\pgfpoint{\pgf@xa + (\pgf@xb - \pgf@xa)/2}{\pgf@yb}}% Upper middle
        \pgfpathlineto{\pgfpoint{\pgf@xa + (\pgf@xb - \pgf@xa)/2}{\pgf@ya + 0.3*(\pgf@yb - \pgf@ya)}}% Down the middle, to 1/4 of the node height

        % These are percentages of the node height.
        % At each percentage/fraction of the height, draw a centered line with a width of a certain percentage (fractionlength) of the overall node width
        \foreach \fractionheight/\fractionlength in {
            0.3/1,%
            0.15/0.75,%
            0/0.5%
        }{
            % One horizontal line on this height:
            \pgfpathmoveto{\pgfpoint{\pgf@xa + (1 - \fractionlength)/2*(\pgf@xb - \pgf@xa)}{\pgf@ya + \fractionheight*(\pgf@yb - \pgf@ya)}}
            \pgfpathlineto{\pgfpoint{\pgf@xb - (1 - \fractionlength)/2*(\pgf@xb - \pgf@xa)}{\pgf@ya + \fractionheight*(\pgf@yb - \pgf@ya)}}
        }
    }
}

\pgfdeclareshape{levelindicator}{
    \inheritsavedanchors[from=rectangle]% Inherit from rectangle, should be good enough. Rectangle has to 'saved anchors': southwest and northeast
    \inheritanchorborder[from=rectangle]
    % Doesn't hurt to also inherit specific anchors:
    \inheritanchor[from=rectangle]{north}
    \inheritanchor[from=rectangle]{north west}
    \inheritanchor[from=rectangle]{north east}
    % \inheritanchor[from=rectangle]{center}
    \inheritanchor[from=rectangle]{west}
    \inheritanchor[from=rectangle]{east}
    \inheritanchor[from=rectangle]{south}
    \inheritanchor[from=rectangle]{south west}
    \inheritanchor[from=rectangle]{south east}

    % When no other anchor is specified, nodes are placed by their 'center' anchor.
    % Change it here to be the lower pointy triangle bit, as opposed to the rectangle center.
    % That way, the triangle always points onto the fluid surface if placed as a node on a drawn line representing this surface
    \anchor{center}{%
        \southwest \pgf@xa=\pgf@x
        \northeast \pgf@xb=\pgf@x \pgf@yb=\pgf@y
        % Lower pointy bit of triangle, as below:
        \pgfpoint{\pgf@xa + (\pgf@xb - \pgf@xa)/2}{\pgf@yb - sqrt(3)*((\pgf@xb - \pgf@xa)/2)}
    }

    \backgroundpath{
        % Store lower left in xa/ya and upper right in xb/yb
        \southwest \pgf@xa=\pgf@x \pgf@ya=\pgf@y
        \northeast \pgf@xb=\pgf@x \pgf@yb=\pgf@y

        \pgfpathmoveto{\pgfpoint{\pgf@xa}{\pgf@yb}}% Upper left corner
        \pgfpathlineto{\pgfpoint{\pgf@xb}{\pgf@yb}}% Upper right corner
        \pgfpathlineto{\pgfpoint{\pgf@xa + (\pgf@xb - \pgf@xa)/2}{\pgf@yb - sqrt(3)*((\pgf@xb - \pgf@xa)/2)}}% Forming an equilateral triangle
        \pgfpathclose

        % These are percentages of the node height.
        % At each percentage/fraction of the height, draw a centered line with a width of a certain percentage (fractionlength) of the overall node width
        \foreach \fractionheight/\fractionlength in {
            0.3/1,%
            0.15/0.75,%
            0/0.5%
        }{
            % One horizontal line on this height:
            \pgfpathmoveto{\pgfpoint{\pgf@xa + (1 - \fractionlength)/2*(\pgf@xb - \pgf@xa)}{\pgf@ya + \fractionheight*(\pgf@yb - \pgf@ya)}}
            \pgfpathlineto{\pgfpoint{\pgf@xb - (1 - \fractionlength)/2*(\pgf@xb - \pgf@xa)}{\pgf@ya + \fractionheight*(\pgf@yb - \pgf@ya)}}
        }
    }
}
