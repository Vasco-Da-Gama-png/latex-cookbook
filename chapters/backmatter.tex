% For the order of things here (appendix, indices, bibliographies, ...) and their
% numbering, see also https://tex.stackexchange.com/a/20547/120853.
% Feel free to deviate since this is a subjective topic.

\appendix

\chapter{An appendix chapter}

Some appendix content can go here, for example detailed computations.

\section{More appendix}

The usual sectioning commands keep working, as does page numbering, showcased by the
following blind text.

\Blindtext

\section{Unprocessed data}
\label{ch:appendix_table}

This is also a good place to put unprocessed data, like \cref{tab:random_long}.

% All following long tables will have the following setting, which could make sense for
% the appendix.
\AtBeginEnvironment{longtabu}{\footnotesize}{}{}

\begin{longtabu}{%
    % ATTENTION: the `tabu` package is unmaintained and a bunch of stuff doesn't work.
    % See also https://github.com/tabu-issues-for-future-maintainer/tabu.
    % Pretty bad situation but works okay enough, and we already use `tabu`/`longtable`
    % anyway for the glossaries in `glossaries-extra`.
    %
    @{}% Remove superfluous left horizontal space
    % NOTE: `X[c]{S}` columns, like in the `tabu` docs, do not really work, see also
    % https://tex.stackexchange.com/q/70279/120853
    S% `S` expects what `\num` would: a number
    s% `s` expects what `\si` would: a unit
    X[2, l]% double the width factor when computing automatic widths
    X[1, l, $]% Dollar sign breaks my editor's syntax highlighting
    @{}% Remove superfluous right horizontal space
}%
    \caption[A \texttt{longtable} from the \ctanpackage{tabu} package]{
        A \texttt{longtable} from the \ctanpackage{tabu} package.
        It can break across pages but still have a caption.
        The \ctanpackage{tabu} package allows automatic width scaling:
        the table is as wide as the text automatically (which sometimes looks bad).
        There are also automatic numbers (note the horizontal decimal point alignment), units (also from \ctanpackage{siunitx}, not \ctanpackage{tabu}) and math columns.
        Note how the table rows are generated automatically and randomly, using Lua%
        \label{tab:random_long}%
    }\\
    \toprule
        Magnitude & Unit & Double-width text & \text{Math column} \\
    \midrule
        118.135 & \nano\meter & Foo & f-x=h\\
        % The following rows look just like that but are generated:
        \directlua{%
            N_ROWS = 75
            dofile("lib/lua/generate_random_table_rows.lua")
        }
    \bottomrule
\end{longtabu}

\chapter{Another appendix chapter}

More stuff can go into the next appendix chapter, for example algorithms and code.

\backmatter

%%%%%%%%%%%%%%%%%%%%%%%%%%%%%%%%%%%%%%%%%%%%%%%%%%%%%%%%
% Bibliography
%%%%%%%%%%%%%%%%%%%%%%%%%%%%%%%%%%%%%%%%%%%%%%%%%%%%%%%%
\nocite{*}% Add everything that has not yet been cited

\label{ch:bibliography}
\printbibliography[% Print the 'real' bibliography of actually cited references
    category=cited,%
]

\printbibliography[%
    notcategory=cited,%
    heading=notcited,%
    env=bibnonum,%
    prenote=further,%
]%

%%%%%%%%%%%%%%%%%%%%%%%%%%%%%%%%%%%%%%%%%%%%%%%%%%%%%%%%
% Index
%%%%%%%%%%%%%%%%%%%%%%%%%%%%%%%%%%%%%%%%%%%%%%%%%%%%%%%%
\setglossarysection{chapter}
\printunsrtindex[%
    style=bookindex,%
]
