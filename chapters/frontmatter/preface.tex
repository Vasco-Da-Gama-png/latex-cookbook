\addchap{Preface}

This document is not a beginner guide.
There is a wide choice of those out there already, both free and paid for.
However, what is lacking is a collection of modern, or even at least current,
best practices.
If you scouted through package documentations and
\href{https://tex.stackexchange.com/}{StackExchange}
long enough, you would eventually get an idea of what is current, idiomatic \LaTeX{}
and what is not.
This is how I learned \LaTeX{} too, so I cannot recommend any useful beginner books.
You might want to check out \href{https://www.overleaf.com/}{Overleaf} though,
an online \LaTeX{} editor%
\footnote{%
    I do not recommend using online editors.
    You are putting your hard work onto some remote, foreign server, relying on
    their ongoing availability and forfeiting the chance of understanding \LaTeX{},
    in case you have to continue locally.
    Theses containing confidential material should also not be hosted externally.
}
with a large knowledge database.

This document is an attempt at collecting best practices
--- or at least, useful approaches ---
and pointing out the old ones they could, and often should, replace.
More than most other languages, the \LaTeX{} code in circulation world\-/wide is
quite aged.
While that code does not necessarily get \emph{worse}, it also does not exactly
age like cheese and wine would.
To that end, notice how the packages integral to this document are actively
maintained and kept modern.

\paragraph{Usage as a Cookbook}
This document is supposed to be used in a \emph{cookbook} style.
That is, it is not meant to be read cover to cover
(declaring it a cookbook is also a useful excuse for the mess I made).

The document contains various \enquote{recipes}, most of which exist in parallel and
are independent of one another.
The goal is to answer questions of the form \emph{How can this and that be done?}
A quick search in the (printed) output or the source code is supposed to deliver
answers.

\paragraph{Usage as a Template}
Considerable effort went into the class file, aka the template.
It pulls all settings together, producing an output that is very different from
vanilla \LaTeX{}.
Therefore, feel free to rip out all the cookbook content, keeping just the settings
files to be used as a template.

\paragraph{Source Code}
This document is meant to be read side\-/by\-/side with its source code.
That is why there is almost no source code in the printed output itself.
If you are curious how a certain output is achieved, navigate to the source code
itself.
This approach was chosen since, plainly, code does not lie, while annotations and
comments might.
So while the printed output will always remain true to its actual source code,
duplicating that source code so it can be read in the printed output directly
is just another vector for errors to creep in.

\paragraph{Source Repository}
The source repository for this document is at
\begin{center}
    \url{https://collaborating.tuhh.de/alex/latex-git-cookbook} .
\end{center}
Any references to the \enquote{source} or \enquote{repository} refer to that project.
