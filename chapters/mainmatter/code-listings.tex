\chapter{Code Listings}
\label{ch:code-listings}

To properly typeset code in \LaTeX{}, we use the package \ctanpackage{listings}.
There is a much more powerful alternative in \ctanpackage{minted}.
But with great power comes great dependencies: \ctanpackage{minted} relies on
Python, and calls in outside help for syntax highlighting using Python's
\texttt{pygments} package.
While that is a great package, the process requires \texttt{--shell-escape} to
compile, and of course Python.
For broad usage and compatibility, that is not suitable.
Lastly, what really broke the deal (for now), is that \ctanpackage{minted} is
incompatible with \ctanpackage{floatrow}, which we use a lot \autocite{egreg_minted_2017}.
Therefore, \ctanpackage{listings} it is.

However, the latter does not come with rich support for either Python or
Modelica.
Support for Python 3 syntax and its numerous built-ins was therefore added manually.
\Citeauthor{winkler_modelica-toolslistings-modelica_2020} provide a configuration
for Modelica syntax highlighting \autocite{winkler_modelica-toolslistings-modelica_2020}.
Lastly, the built-in support of MATLAB was manually extended to include all
(as of version 2020a) over 2500 base functions, as listed by \citeauthor{mathworks_matlab_2020}
\autocite{mathworks_matlab_2020}.
It also includes all \emph{keywords} as returned by running \texttt{iskeyword} in
the MATLAB prompt.

The available languages have to be given as environment options to the \texttt{lstlisting}
environment in the form: \texttt{language={[<dialect>]<language>}}.
Note the braces around the argument if a \emph{dialect} is used, which is required for
escaping the brackets.
Since \ctanpackage{listings} already defines Python and MATLAB, but not Modelica, the
new capabilities are only available as a dialect for the first two:
\begin{description}
    \item[Python:] \texttt{language=\{[3]Python\}},
    \item[MATLAB:] \texttt{language=\{[Custom]MATLAB\}},
    \item[Modelica:] \texttt{language=Modelica}.
\end{description}

\paragraph{Color Scheme}
Note how the color scheme is the same throughout languages.
The idea is that keywords of similar importance or status are highlighted uniformly.
For example, there are keywords responsible for class and function definitions in most
(all?) object\-/oriented languages.
Another example are error\-/handling and -throwing keywords, which many languages
offer.
All these different types should be identified and treated equally.
For this template, the groundwork for the three \enquote{new} languages is already done.

If the current colors are not to your liking (that is, remind you of an
\emph{angry fruit salad}) they can easily be changed.
This is possible without having to deal with the mapping of keywords (for example
\lstinline[language={[3]Python}]|class| for Python or
\lstinline[language={[Custom]Matlab}]|classdef| for MATLAB)%
\footnote{%
    Notice how these keywords were created using an \emph{inline} listing, like:
    \lstinline[language={[3]Python}]|y = [file_patterns[x] for x in ["send", "help"]]|.
}
into types (in this case, some sort of \enquote{class definition}\-/type).
The \ctanpackage{listings} package does this numerically and just numbers groups of
keywords, so there is no need to name them.
These keyword groups only change when the language itself changes.

\paragraph{References}
Individual code lines can also be referenced.
For example, we find a \lstinline[language={[3]Python}]|return| statement on
\cref{line:python_return} in \cref{lst:float_example}.

The following examples are not always complete or functioning, they are only supposed
to showcase the available syntax highlighting.

\section{Python}

The demonstrations in this chapter are done using Python, since the modifications to
\ctanpackage{listings} were focused on that.
There are examples for the other two mentioned languages later.
One feature is a code snippet style called \texttt{betweenpar}, looking like:
\begin{lstlisting}[%
    style=betweenpar,%
]
    def get_nonempty_line(
        lines: Iterable[str],
        last: bool = True
    ) -> str:
        if last:
            lines = reversed(lines)
        return next(line for line in lines if line.rstrip())
\end{lstlisting}
It is intended for small samples of code that flow into the surrounding text.
A second feature are code listings as regular floats, as demonstrated in
\cref{lst:float_example}.
As floats, they behave like any other figure, table \iecfeg{etc}.

\begin{lstlisting}[
    float,
    caption={%
        This is a caption.
        Listings cannot be overly long since floats do not page-break% No period here!
    },
    label={lst:float_example},
]
    import json
    import logging.config
    from pathlib import Path

    from resources.helpers import path_relative_to_caller_file

    ¬\phstring{\LaTeX{} can go in here: \(\sum_{i = 1}^{n} a + \frac{\pi}{2} \)}¬


    def set_up_logging(logger_name: str) -> logging.Logger:
        """Set up a logging configuration."""
        config_filepath = path_relative_to_caller_file("logger.json")  # same directory

        try:
            with open(config_filepath) as config_file:
                config: dict = json.load(config_file)
            logging.config.dictConfig(config)
        except FileNotFoundError:
            logging.basicConfig(
                level=logging.INFO, format="[%(asctime)s: %(levelname)s] %(message)s"
            )
            logging.warning(f"Using fallback: no logging config found at {config_filepath}")
            logger_name = __name__

        return logging.getLogger(logger_name) ¬\label{line:python_return}¬
\end{lstlisting}

Lastly, you can use the base \texttt{lstlisting} environment for more elaborate,
possibly very long code listings.
These can be broken across pages and are probably best suited for the appendix.
\begin{lstlisting}
    def ansi_escaped_string( ¬\phnote{A random reference: \cref{fig:censorbox}}¬
        string: str,
        *,
        effects: Union[List[str], None] = None,
        foreground_color: Union[str, None] = None,
        background_color: Union[str, None] = None,
        bright_fg: bool = False,
        bright_bg: bool = False,
    ) -> str:
        """Provides a human-readable interface to escape strings for terminal output.
    
        Using ANSI escape characters, the appearance of terminal output can be changed. The
        escape chracters are numerical and impossible to remember. Also, they require
        special starting and ending sequences. This function makes accessing that easier.
    
        Args:
            string: The input string to be escaped and altered.
            effects: The different effects to apply, e.g. underlined.
            foreground_color: The foreground, that is text color.
            background_color: The background color (appears as a colored block).
            bright_fg: Toggle whatever color was given for the foreground to be bright.
            bright_bg: Toggle whatever color was given for the background to be bright.
    
        Returns:
            A string with requested ANSI escape characters inserted around the input string.
        """
    
        def pad_sgr_sequence(sgr_sequence: str = "") -> str:
            """Pads an SGR sequence with starting and end parts.
    
            To 'Select Graphic Rendition' (SGR) to set the appearance of the following
            terminal output, the CSI is called as:
            CSI n m
            So, 'm' is the character ending the sequence. 'n' is a string of parameters, see
            dict below.
    
            Args:
                sgr_sequence: Sequence of SGR codes to be padded.
            Returns:
                Padded SGR sequence.
            """
            control_sequence_introducer = "\x1B["  # hexadecimal '1B'
            select_graphic_rendition_end = "m"  # Ending character for SGR
            return control_sequence_introducer + sgr_sequence + select_graphic_rendition_end
    
        # Implement more as required, see
        # https://en.wikipedia.org/wiki/ANSI_escape_code#SGR_parameters.
        sgr_parameters = {
            "underlined": 4,
        }
    
        sgr_foregrounds = {  # Base hardcoded mapping, all others can be derived
            "black": 30,
            "red": 31,
            "green": 32,
            "yellow": 33,
            "blue": 34, ¬\phnum{\(30 + 4\)}¬
            "magenta": 35,
            "cyan": 36,
            "white": 37,
        }
    
        # These offsets convert foreground colors to background or bright color codes, see
        # https://en.wikipedia.org/wiki/ANSI_escape_code#Colors
        bright_offset = 60
        background_offset = 10
    
        if bright_fg:
            sgr_foregrounds = {
                color: value + bright_offset for color, value in sgr_foregrounds.items()
            }
    
        if bright_bg:
            background_offset += bright_offset
    
        sgr_backgrounds = {
            color: value + background_offset for color, value in sgr_foregrounds.items()
        }
    
        # Chain various parameters, e.g. 'ESC[30;47m' to get white on black, if 30 and 47
        # were requested. Note, no ending semicolon. Collect codes in a list first.
        sgr_sequence_elements: List[int] = []
    
        if effects is not None:
            for sgr_effect in effects:
                try:
                    sgr_sequence_elements.append(sgr_parameters[sgr_effect])
                except KeyError:
                    raise NotImplementedError(
                        f"Requested effect '{sgr_effect}' not available."
                    )
        if foreground_color is not None:
            try:
                sgr_sequence_elements.append(sgr_foregrounds[foreground_color])
            except KeyError:
                raise NotImplementedError(
                    f"Requested foreground color '{foreground_color}' not available."
                )
        if background_color is not None:
            try:
                sgr_sequence_elements.append(sgr_backgrounds[background_color])
            except KeyError:
                raise NotImplementedError(
                    f"Requested background color '{background_color}' not available."
                )
    
        # To .join() list, all elements need to be strings
        sgr_sequence: str = ";".join(str(sgr_code) for sgr_code in sgr_sequence_elements)
    
        reset_all_sgr = pad_sgr_sequence()  # Without parameters: reset all attributes
        sgr_start = pad_sgr_sequence(sgr_sequence)
        return sgr_start + string + reset_all_sgr
\end{lstlisting}


\section{MATLAB}

This section contains example code for MATLAB, for example in the following,
\texttt{betweenpar}\-/styled block.

\begin{lstlisting}[%
    style=betweenpar,%
    language={[Custom]Matlab},%
]
    %{
        Universal Gas Constant for SIMULINK.
    %}
    R_m = Simulink.Parameter;
        R_m.Value = 8.3144598;
        R_m.Description = 'universal gas constant';
        R_m.DocUnits = 'J/(mol*K)';
\end{lstlisting}

\begin{lstlisting}[%
    float,%
    language={[Custom]Matlab},%
    caption={%
        [A class definition in MATLAB]%
        A class definition in MATLAB, from \cite{mathworks_create_2020}%
    },%
    label={lst:matlab_class_definition},%
]
    classdef BasicClass
        properties
            Value {mustBeNumeric}
        end
        methods
            function r = roundOff(obj)
                r = round([obj.Value],2);
            end
            function r = multiplyBy(obj,n)
                r = [obj.Value] * n;
            end
        end
    end
\end{lstlisting}

Of course, floats (see \cref{lst:matlab_class_definition}) are available as well.
So are longer sections:

\begin{lstlisting}[
    language={[Custom]Matlab}
]
    fullfilepath = mfilename('fullpath');
    [filepath, filename, ~] = fileparts(fullfilepath);

    %% Begin Dialogue
    %{
    Extract Data from user-specified input. Can be either a File that is run
    (an *.m-file), variables in the Base Workspace or existing data from a
    previous run (a *.MAT-file). All variables are expected to be in Table
    format, which is the data type best suited for this work. Therefore, we
    force it. No funny business with dynamically named variables or naked
    matrices allowed.
    %}
    pass_data = questdlg({'Would you like to pass existing machine data?', ...
        'Its data would be used to compute your request.', ...
        ['Regardless, note that data is expected to be in ',...
        'Tables named ''', dflt.comp, ''' and ''', dflt.turb, '''.'], '',...
        'If you choose no, existing values are used.'}, ...
        'Machine Data Prompt', btnFF, btnWS, btnNo, btnFF);

    switch pass_data
        case btnFF
            prompt = {'File name containing the Tables:', ...
                'Compressor Table name in that file:',...
                'Turbine Table name in that file:'};
            title = 'Machine Data Input';
            dims = [1 50];
            definput = {dflt.filename.data, dflt.comp, dflt.turb};
            machine_data_file = inputdlg(prompt, title, dims, definput);
            if isempty(machine_data_file)
                fprintf('[%s] You left the dialogue and function.\n',...
                    datestr(now));
                return
            end
            run(machine_data_file{1});%spills file contents into funcWS
        case btnWS
            prompt = {'Base Workspace Compressor Table name:',...
                    'Base Workspace Turbine Table name:'};
            title = 'Machine Data Input';
            dims = [1 50];
            definput = {dflt.comp, dflt.turb};
            machine_data_ws = inputdlg(prompt, title, dims, definput);
            if isempty(machine_data_ws)
                fprintf('[%s] You left the dialogue and function.\n',...
                    datestr(now));
                return
            end
        case btnNo
            boxNo = msgbox(['Looking for stats in ''', dflt.filename.stats, ...
                '''.'], 'Using Existing Data', 'help');
            waitfor(boxNo);
            try
                stats = load(dflt.filename.stats);
                stats = stats.stats;
            catch ME
                switch ME.identifier
                    case 'MATLAB:load:couldNotReadFile'
                        warning(['File ''', dflt.filename.stats, ''' not ',...
                            'found in search path. Make sure it has been ',...
                            'generated in a previous run or created ',...
                            'manually. Resorting to hard-coded data ',...
                            'for now.']);
                            a_b = 200.89;
                            x_c = 0.0012;
                            f_g = 10.0;
                    otherwise
                        rethrow(ME);
                end
            end
        case ''
            fprintf('[%s] You left the dialogue and function.\n',datestr(now));
            return
        otherwise%Only gets here if buttons are misconfigured
            error('This option is not coded, should not get here.');
    end
\end{lstlisting}

\subsection{MATLAB/Simulink icons}

For an older project, MATLAB/Simulink vector icons were created.
They are included here at the off chance that someone else might find a use for these.
\begin{itemize}
    \item \mtlbsmlkicon{matlab_object_box}
    \item \mtlbsmlkicon{matlab_struct}
    \item \mtlbsmlkicon{matlab_table}
    \item \mtlbsmlkicon{simulink_algebraic_constraint}
    \item \mtlbsmlkicon{simulink_base_workspace}
    \item \mtlbsmlkicon{simulink_configuration}
    \item \mtlbsmlkicon{simulink_data_dictionary}
    \item \mtlbsmlkicon{simulink_library_model}
    \item \mtlbsmlkicon{simulink_library}
    \item \mtlbsmlkicon{simulink_log_data}
    \item \mtlbsmlkicon{simulink_lut}
    \item \mtlbsmlkicon{simulink_model_workspace}
    \item \mtlbsmlkicon{simulink_model}
    \item \mtlbsmlkicon{simulink_referenced_model}
    \item \mtlbsmlkicon{simulink_step_block}
\end{itemize}

\section{Modelica}

Naturally, floating and all other environments and styles are also available for
Modelica.
The syntax highlighting for a few basic code samples \autocite{wikipedia_modelica_2020} looks like:

\begin{lstlisting}[
    language=Modelica,%
]
    x := 2 + y;
    x + y = 3 * z;

    model FirstOrder
        parameter Real c=1 "Time constant";
        Real x (start=10) "An unknown";
    equation
        der(x) = -c*x "A first order differential equation";
    end FirstOrder;

    type Voltage = Real(quantity="ElectricalPotential", unit="V");
    type Current = Real(quantity="ElectricalCurrent", unit="A");

    connector Pin "Electrical pin"
        Voltage      v "Potential at the pin";
        flow Current i "Current flowing into the component";
    end Pin;

    model Capacitor
        parameter Capacitance C;
        Voltage u "Voltage drop between pin_p and pin_n";
        Pin pin_p, pin_n;
    equation
        0 = pin_p.i + pin_n.i;
        u = pin_p.v - pin_n.v;
        C * der(u) = pin_p.i;
    end Capacitor;

    model SignalVoltage
        "Generic voltage source using the input signal as source voltage"
        Interfaces.PositivePin p;
        Interfaces.NegativePin n;
        Modelica.Blocks.Interfaces.RealInput v(unit="V") 
            "Voltage between pin p and n (= p.v - n.v) as input signal";
        SI.Current i "Current flowing from pin p to pin n";
    equation
        v = p.v - n.v;
        0 = p.i + n.i;
        i = p.i;
    end SignalVoltage;

    model Circuit
        Capacitor C1(C=1e-4) "A Capacitor instance from the model above";
        Capacitor C2(C=1e-5) "A Capacitor instance from the model above";
            ...
    equation
        connect(C1.pin_p, C2.pin_n);
            ...
    end Circuit;
\end{lstlisting}
