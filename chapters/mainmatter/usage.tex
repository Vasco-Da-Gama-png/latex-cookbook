\chapter{Usage}

This document \enquote{requires}%
\footnote{
    It does not \emph{technically} require Docker, but I hope to convince you that
    the non\-/Docker, manual way is the way of the dodo and a big no\-/no.
}
\href{https://www.docker.com/}{Docker}.
On a high level, Docker allows to prepare specific software bundles, tailor\-/made for
whatever application the bundle author desires.
Other users can then use these software bundles for their own projects.
The bundles are well\-/defined and can be distributed and run very easily.
The more complex and demanding the required software, the higher the benefit of using such bundles.
The driving design principle and primary use case for these bundles is \emph{isolation}.
Whatever you do with the bundles, it happens in isolation from your host system.
This allows a user to have, for example, arbitrary Python versions at their disposal,
whereas a local system installation can only ever offer a single version.

For a \LaTeX{} document, the one at hand here is pretty complex.
This is owed to the many used packages and outside tooling.
Outside tooling (programs other than \LaTeX{} called from within \LaTeX{}) is quite
prevalent in \LaTeX{}, since it itself is so limited.%
\footnote{
    For example, when writing Python, you would not call Perl or JavaScript from within it,
    because whatever they can do, Python can.
    The same analogy does \emph{not} hold for \LaTeX{}: base \LaTeX{} can do surprisingly
    little, even though \TeX{} is technically Turing\-/complete.
}
Let us look at the outside tooling used for this document.

\section{Outside tools and special packages}

This section highlights the pain points of \emph{not} using Docker.
If you are already familiar and need no convincing, skip to \cref{ch:using-docker}.
For this document, outside tools are required for:
\begin{enumerate}
    \item \ctanpackage{glossaries-extra}: requires the outside tool
        \ctanpackage{bib2gls}, see \cref{ch:bib2gls}, which in turn requires
        a Java Runtime Environment.
    \item \ctanpackage{biblatex}: requires the outside tool \ctanpackage{biber},
        see \cref{ch:bibliography_rationale}.
    \item \ctanpackage{svg}: requires the outside program
        \href{https://inkscape.org/}{InkScape}.
        Examples for that package's main command, \verb|\includesvg|, are
        \cref{fig:wide_caption,fig:tighter_caption,fig:multiple_floats,fig:sidecap,fig:inside_float}.
    \item Using \texttt{contour gnuplot} commands with \ctanpackage{pgfplots}, see
        \cref{fig:mollier_diagram}, requires
        \href{http://www.gnuplot.info/download.html}{gnuplot}.
    \item Syntax highlighting for source code, see \cref{ch:code-listings},
        is done through \ctanpackage{minted}.
        It requires \href{https://www.python.org/}{Python} (built into virtually all Linux
        distributions but needs to be installed on Windows) with its
        \href{https://pypi.org/project/Pygments/}{pygments} package installed.
        That package does the heavy lifting and \ctanpackage{minted} inserts the result
        into your \LaTeX{} document.
\end{enumerate}

If you have a proper \LaTeX{} distribution, \texttt{bib2gls} and \texttt{biber}
will already be available as commands and ready to be invoked.
The latter means that they are on your \texttt{\$PATH}, \iecfeg{i.e.}\ the paths to
the respective binaries are part of the \texttt{PATH} environment variable.
\emph{All other stuff, you have to install manually}.

You will have to install and keep everything else updated by hand.
What if one of the dependencies of this document conflicts with something you have already
installed on your system?
What if that conflict is not resolvable?
You would be unable to use this document.
This is where Docker comes into play.
You outsource all this setting\-/up to someone who already bundled all that stuff together.
Docker calls these \enquote{bundles} \emph{images}.
There is a Docker image \emph{tailor\-/made} for this document:
\begin{center}
    \href{https://hub.docker.com/r/alexpovel/latex}{alexpovel/latex}.
\end{center}
It is guaranteed to function correctly for this document, since the author maintains both
in parallel.
There will never be a mismatch.
If maintenance ceases, it will cease for both components at the same time, hence it will
still continue to work, just not get updated anymore.

\section{Using Docker}
\label{ch:using-docker}

Instead of all of the above, \textbf{only one installation is required}: Docker.
You can get it from
\begin{center}
    \url{https://docs.docker.com/get-docker/} .
\end{center}
\emph{You don't even need a \LaTeX{} distribution}.
All you need is an editor to edit the files in, see \cref{ch:editor}.
Once you want to compile, open a terminal, navigate to your directory and run
\begin{minted}[linenos=false]{powershell}
    docker run --rm --volume ${PWD}:/tex alexpovel/latex
\end{minted}
for PowerShell or
\begin{minted}[linenos=false]{shell}
    docker run --rm --volume $(pwd):/tex alexpovel/latex
\end{minted}
for bash.
That's it!

The command consists of:
\begin{itemize}
    \item The \texttt{--rm} option, which removes the run container after you are done
        (containers are \enquote{instances} of images).
        This is generally desired, since containers are ephemeral and should be treated as such.
        Do not keep containers around, simply create new ones!
        If you need adjustments, adjust the image, not the container, then create new containers
        from that adjusted image.
    \item The \texttt{--volume} option makes the current directory (\texttt{pwd})
        available \emph{inside} the running container, at the \texttt{/tex} destination path.
        The \LaTeX{} process needs your files to work with.
        Without this option, the container would be \enquote{naked}, with no way of
        accessing your files.
    \item The final argument to \texttt{docker run} is the image to be run, in this case
        \texttt{alexpovel/latex}, which will look for that name on
        \href{https://hub.docker.com/}{DockerHub}.
\end{itemize}

Ideally, the command can be registered as the \enquote{compilation} command of your
editor.
That way, you just hit the compile button and will be using Docker in the background,
with no changes to your workflow.

\subsection{Compilation steps}

You are probably used to running \texttt{pdflatex} or similar on your source files,
as many times as needed.
So where does that step happen in the above \texttt{docker} command?

The Docker approach uses the \textbf{\texttt{latexmk}} tool to ease all the painful labour
of running chains of \texttt{pdflatex}, \texttt{biber} \iecfeg{etc}.\ manually.
\texttt{latexmk} automates \LaTeX{} compilation by detecting all the required
steps and then running them as often as required.
It requires \textbf{Perl}.
Linux users will already have it available, Windows users may grab
\href{http://strawberryperl.com/}{Strawberry Perl}.
As such, this document's processing pipeline \emph{as a whole} requires Perl,
although it is technically not required for document compilation only.

Once Perl is installed (of course, the Docker image already contains it),
the entire document can be compiled by \textbf{simply calling \texttt{latexmk}}.
You do not even have to provide a \texttt{*.tex} file argument.
By default, \texttt{latexmk} will simply compile all found \texttt{*.tex} files.
The core ingredient to this magic process is the \texttt{.latexmkrc} configuration file.
You can find it in the repository root directory.
It is tailored to this document and does not need to be touched if the compilation
process itself has not changed.
It also contains some more insights to the entire process.

\texttt{latexmk} is great because it figures out most things by itself and enjoys
wide\-/spread acceptance and adoption.
If it does not figure out everything from the get\-/go, it is easily customized,
like for this document.

Having walked through all this manually, hopefully using the prepared Docker image
instead makes more sense now.
It is guaranteed to work for everyone, because the Docker container (that is, the
virtual build environment) will be identical for all users.
It is independent of local \LaTeX{} installations and all their quirks.
As such, it simply and forever does away with the entire, huge class of
\begin{displayquote}[Everyone, at some point]
    But it works on my machine!
\end{displayquote}
Good riddance to that.

If all of this is embedded into a pipeline on GitLab, your documents are built whenever
you \texttt{git push} to the remote server (or whenever you configure it to).
It does not get simpler; the downside is of course the lengthier setup, but all of that
is explained in the
\href{https://collaborating.tuhh.de/alex/latex-git-cookbook/-/blob/master/README.md}{README}.
Also, the repository itself is a live demonstration where everything is set up already!

\subsection{More on Docker}

You do not need to know of the entire chain of how Docker images are created and run.
Only consuming the final image has all the benefits with little effort.
However, the process is not complex:
\begin{enumerate}
    \item A \textbf{\texttt{Dockerfile} text document} is created, containing instructions on how the image should look like (like what stuff to install, what to copy where, ...).

        As a baseline, these instructions often rely on a Debian distribution.
        As such, all the usual Debian/Linux tools can be accessed, like \texttt{bash}.

        An (unrelated)
        \href{https://github.com/alexpovel/random_python/blob/master/music-converter/Dockerfile}{example Dockerfile}
        can look like:

        \begin{minted}{dockerfile}
# Get the latest Debian Slim with Python installed
FROM python:slim

# Update the Debian package repositories and install a Debian package.
# Agree to installation automatically (`-y`)!
# This is required because Dockerfiles need to run without user interaction.
RUN apt-get update && apt-get install -y ffmpeg

# Copy a file from the building host into the image
COPY requirements.txt .

# Run some shell command, as you would in a normal sh/bash environment.
# This is a Python-specific command to install Python packages according to some
# requirements.
RUN pip install -r requirements.txt

# Copy more stuff!
COPY music-converter/ music-converter/

# This will be the command the image executes if run.
# It runs this command as a process and terminates as soon as the process ends
# (successfully or otherwise).
# Docker is not like a virtual machine: it is intended to run *one* process, then
# die. If you need to run it again, just create a new container (instance of a
# Docker image). Treat containers as *cattle*, not as a *pet*. The
# container-recreation process is light-weight, fast and the way to go.
#
# Of course, this does not stop anyone from running one *long-running* process
# (as in infinity, `while True`-style). This is still a good use-case for Docker
# (as are most things!). An example for this is a webserver.
ENTRYPOINT [ "python", "-m", "music-converter", "/in", "--destination", "/out" ]
        \end{minted}

        The Dockerfile this project uses for LaTeX stuff is here:
        \begin{center}
            \url{https://github.com/alexpovel/latex-extras-docker/blob/master/Dockerfile}
        \end{center}
        It is not as simple, so not as suited for an example.
        Its length gives you an idea of the setup required to compile this \LaTeX{} document.
        All of that complexity is of no concern to you when using Docker!
        Of course, such an image also works for much simpler documents.

        If you require custom additions, you can always inherit from existing base images:
        \begin{minted}{dockerfile}
FROM alexpovel/latex

# ... Your stuff goes here ...
        \end{minted}
    \item The \textbf{image} is then built according to the \texttt{Dockerfile} instructions,
        resulting in a large\-/ish file that contains an executable environment.
        For example, if we install a comprehensive TeXLive distribution, the image can be
        more than \SI{2}{\giga\byte} in size.
        Note that you will never interact with that \enquote{file} directly.
        Docker manages it for you, and all interaction occurs through the \texttt{docker} command.

        The Docker image can be distributed.
        If you just instruct to run an image called \iecfeg{e.g.} \texttt{alexpovel/latex}, without
        specifying a full URL to somewhere, Docker will look on its Hub for an image of that
        name (and find it \href{https://hub.docker.com/r/alexpovel/latex}{here}).
        Anyone can pull (public) images from there, and everyone will
        be on the same page (alternatively, you can build the image from the Dockerfile).

        For example, as stated, the \LaTeX{} environment for this project requires a whole bunch of setting\-/up.
        This can take more than an afternoon to read up upon, understand, implement and getting to run.
        In some cases, it will be impossible if some required part of a project conflicts
        with a pre\-/existing condition on your machine.
        For example, suppose project \emph{A} requires Perl in version \texttt{6.9.0},
        but project \emph{B} requires version \texttt{4.2.0}.
        This is what Docker is all about: isolation.
        Whatever is present on your system does not matter, only the Docker image/container
        contents are relevant.

        Further, if you for example specify \mintinline{dockerfile}{FROM python:3.8.6}
        as your base image, aka provided a so\-/called tag of \texttt{3.8.6}, it will be that tag in ten years' time still.
        As such, you nailed the version your process takes place in and requires.
        Once set up, this will run on virtually any computer running Docker, be it your
        laptop now or whatever your machine is in ten years.
        This is especially important for the reproducibility of research.
    \item Once the image is created, it can be run, \textbf{creating a container}.
        We can then enter the container and use it like a pretty normal (usually Linux)
        machine, for example to compile our \LaTeX{} files.
        Other, single commands can also be executed.

        The proper way is to run one container \emph{per process}.
        If that process (\iecfeg{e.g.}\ \texttt{latexmk}) finishes, the container exits.
        A new process then requires a new container.
\end{enumerate}

\subsection{Editor}
\label{ch:editor}

You are free to do whatever you want.
However, a garbage editor can substantially hamstring your work.
For example, please do not use Notepad++.
It is a fantastic little program but unsuitable for any serious, longer work.

The author uses and dearly recommends \href{https://code.visualstudio.com/}{Visual Studio Code},
using its
\href{https://marketplace.visualstudio.com/items?itemName=James-Yu.latex-workshop}{\LaTeX{} Workshop} extension, which provides syntax highlighting, shortcuts and many other useful things.
VSCode is among the most state\-/of\-/the\-/art editors currently available.
Being usable for \LaTeX{} is just a nice \enquote{side\-/effect} we can take
advantage of.
It is open\-/source and therefore also has a privacy\-/respecting alternative fork, \emph{VSCodium}.

For a more conventional, complete \abb{integrated_development_environment},
try \href{https://www.texstudio.org/}{TeXStudio}.
Like VSCode, it is also
\href{https://github.com/texstudio-org/texstudio}{open source}.
TeXStudio will cater to \SI{99}{\percent} of your \LaTeX{} needs, but nothing else (Markdown, \dots{}).
